\chapter[Conclusões Parciais]{Conclusões Parciais}
\label{Conclusao}

Essa primeira etapa do projeto permitiu adquirir conceitos importantes, como noções sobre o funcionamento de um mercado livre de energia e sobre especificidades do funcionamento do PJM; contato com ferramentas de programação, com programação matemática, problemas de otimização e solucionadores comerciais.

A partir do estudo bibliográfico, foi possível entender em linhas gerais como um mercado funciona, suas siglas e os cálculos principais que são feitos para determinar o preço da energia. Além disso, foi possível ganhar conhecimentos específicos, como a possibilidade de existir energia com preço negativo, algo que inspira bastante a ideia de operar um \ac{BESS}.

O trabalho permitiu a ambientação com a forma como os dados eram adquiridos e mostrou-se que é possível automatizar essa tarefa através de ferramentas computacionais que permitem que tarefas iterativas sejam feitas de modo mais rápido e preciso.

De posse dos dados organizados e dispostos no formato DAT, foi possível ganhar noções de programação matemática através do \textit{software} AMPL. Assim, foram feitas otimizações como vistas em \cite{salles2017}, de modo a validar os dados adquiridos, consolidar essa etapa e permitir o início da aplicação das restrições equacionadas em \ref{constrains}.


Por fim, esse projeto permitiu o aprofundamento nos estudos relacionados a utilização de sistemas de armazenamento por meio de baterias operando em mercados livres de energia, sendo um estudo marcado pela adição do equacionamento do envelhecimento da bateria. Houve também importantes avanços na automatização no modo como os dados são adquiridos. Pode-se então colocar em prática os conhecimentos adquiridos nas diferentes matérias da graduação.

