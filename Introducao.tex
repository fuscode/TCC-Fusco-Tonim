\chapter[Introdução]{Introdução}
\label{intro}

O mercado livre de energia no Brasil vem crescendo e amadurecendo ao longo do tempo. Em seus mais de 20 anos de existência, passou por importantes avanços de legislação visando disponibilizar fontes de geração cada vez mais limpas, distribuídas e seguras além de garantir a liberdade de negociação ao consumidor \cite{folha2018ml}. 

Entre as propostas legislativas, é possível identificar algumas aproximações com o mercado norte-americano de energia elétrica \cite{ccee2018}. Atualmente no Brasil, o \ac{PLD} é calculado em uma base semanal pela \ac{CCEE} e é regulamentado pela \ac{ANEEL}. 

Já nos Estados Unidos, em algumas regiões, esse cálculo é feito e disponibilizado em um ritmo muito mais intenso; em mercados como o PJM, o cálculo da tarifa do tipo \textit{Real-Time} é feito a cada 5 minutos \cite{pjmEnergyMarket}. A partir disso, surge um mercado muito mais competitivo e que exige um certo conhecimento e estudo prévio além do desenvolvimento de modelos e algoritmos de cálculos para que as tomadas de decisão nesse novo mercado muito mais veloz sejam bem embasadas.

O PJM é um mercado competitivo e atacadista que gerencia a confiabilidade de sua rede de transmissão, despachando e coordenando o fluxo de potência em sua zona de atuação. Esse mercado inclui a negociação do Mercado do Dia Seguinte ou \ac{DAM} e Mercado em Tempo Real ou \ac{RTM}, armazenamento e serviços auxiliares. \cite{fercPJM}

Traçando um paralelo com o mercado brasileiro, é possível identificar que o mercado norte-americano possui características que estão vindo gradualmente para o país, como, por exemplo, a negociação em tempo real através do cálculo horário da tarifa. Assim, é de grande importância para o mercado livre nacional a assimilação de algoritmos de cálculos baseados em nós da rede e em regime horário de operação e das estratégias de negociação decorrentes.

\section{Objetivos}

Compreender o mercado de energia norte-americano com enfoque no PJM e suas especificidades, como o cálculo do \ac{LMP} que por sua vez reflete diretamente no preço praticado da energia, entendimento de conceitos como \textit{Real-Time Pricing} (Precificação em Tempo Real), \textit{Day-Ahead Market} (Mercado do Dia Seguinte) e o impacto desses fatores no traçado de estratégias de compra, venda e estocagem de energia elétrica para maximização da receita. 

Adotar e implementar na linguagem de programação algébrica AMPL modelos matemáticos que correspondam à baterias de lítio levando em conta seu envelhecimento e desgaste de modo a aperfeiçoar o algoritmo que usava um modelo genérico visto em \cite{salles2017}. 

Finalmente, será feita uma análise dos conjunto de dados obtidos em uma plataforma de \textit{Business Analytics} e os dados serão dispostos de maneira interativa e simplificada para fácil acesso a possíveis investidores e interessados na área. 







